\documentclass[]{article}
\usepackage{lmodern}
\usepackage{amssymb,amsmath}
\usepackage{ifxetex,ifluatex}
\usepackage{fixltx2e} % provides \textsubscript
\ifnum 0\ifxetex 1\fi\ifluatex 1\fi=0 % if pdftex
  \usepackage[T1]{fontenc}
  \usepackage[utf8]{inputenc}
\else % if luatex or xelatex
  \ifxetex
    \usepackage{mathspec}
  \else
    \usepackage{fontspec}
  \fi
  \defaultfontfeatures{Ligatures=TeX,Scale=MatchLowercase}
\fi
% use upquote if available, for straight quotes in verbatim environments
\IfFileExists{upquote.sty}{\usepackage{upquote}}{}
% use microtype if available
\IfFileExists{microtype.sty}{%
\usepackage{microtype}
\UseMicrotypeSet[protrusion]{basicmath} % disable protrusion for tt fonts
}{}
\usepackage[margin=1in]{geometry}
\usepackage{hyperref}
\hypersetup{unicode=true,
            pdftitle={ISLR Chap 3 Notes},
            pdfauthor={M. E. Waggoner},
            pdfborder={0 0 0},
            breaklinks=true}
\urlstyle{same}  % don't use monospace font for urls
\usepackage{graphicx,grffile}
\makeatletter
\def\maxwidth{\ifdim\Gin@nat@width>\linewidth\linewidth\else\Gin@nat@width\fi}
\def\maxheight{\ifdim\Gin@nat@height>\textheight\textheight\else\Gin@nat@height\fi}
\makeatother
% Scale images if necessary, so that they will not overflow the page
% margins by default, and it is still possible to overwrite the defaults
% using explicit options in \includegraphics[width, height, ...]{}
\setkeys{Gin}{width=\maxwidth,height=\maxheight,keepaspectratio}
\IfFileExists{parskip.sty}{%
\usepackage{parskip}
}{% else
\setlength{\parindent}{0pt}
\setlength{\parskip}{6pt plus 2pt minus 1pt}
}
\setlength{\emergencystretch}{3em}  % prevent overfull lines
\providecommand{\tightlist}{%
  \setlength{\itemsep}{0pt}\setlength{\parskip}{0pt}}
\setcounter{secnumdepth}{5}
% Redefines (sub)paragraphs to behave more like sections
\ifx\paragraph\undefined\else
\let\oldparagraph\paragraph
\renewcommand{\paragraph}[1]{\oldparagraph{#1}\mbox{}}
\fi
\ifx\subparagraph\undefined\else
\let\oldsubparagraph\subparagraph
\renewcommand{\subparagraph}[1]{\oldsubparagraph{#1}\mbox{}}
\fi

%%% Use protect on footnotes to avoid problems with footnotes in titles
\let\rmarkdownfootnote\footnote%
\def\footnote{\protect\rmarkdownfootnote}

%%% Change title format to be more compact
\usepackage{titling}

% Create subtitle command for use in maketitle
\newcommand{\subtitle}[1]{
  \posttitle{
    \begin{center}\large#1\end{center}
    }
}

\setlength{\droptitle}{-2em}

  \title{ISLR Chap 3 Notes}
    \pretitle{\vspace{\droptitle}\centering\huge}
  \posttitle{\par}
    \author{M. E. Waggoner}
    \preauthor{\centering\large\emph}
  \postauthor{\par}
      \predate{\centering\large\emph}
  \postdate{\par}
    \date{February 3, 2019}

\newcommand{\rss}{\mathrm{RSS}}
\newcommand{\tss}{\mathrm{TSS}}
\newcommand{\rse}{\mathrm{RSE}}
\newcommand{\mse}{\mathrm{MSE}}
\newcommand{\var}{\mathrm{Var}}
\newcommand{\se}{\mathrm{SE}}
\newcommand{\df}{\mathrm{df}}
\newcommand{\cor}{\mathrm{Cor}}

\begin{document}
\maketitle

{
\setcounter{tocdepth}{2}
\tableofcontents
}
\section{Simple Linear Regression}\label{simple-linear-regression}

Linear vector equation

\begin{align}
Y = \beta_0 + \beta_1 X
\end{align}

Linear prediction of \(y\) based on \(x\)

\begin{align}
\hat{y} = \hat{\beta}_0 + \hat{\beta}_1 x
\end{align}

\subsection{Estimating the
Coefficients}\label{estimating-the-coefficients}

\(i\)th Residual

\begin{align*}
e_i = y_i - \hat{y}_i 
\end{align*}

Residual sum of squares

\begin{align}
    \rss =& e_1^2   + e_2^2 + \cdots + e_n^2 \nonumber \\ 
    =& \left( y_1 - \hat{y}_1\right)^2 + \left( y_2 - \hat{y}_2\right)^2 + \cdots + \left( y_n - \hat{y}_n\right)^2 
\end{align}

Least squares coefficients estimates that minimize RSS where \(\bar{x}\)
and \(\bar{y}\) are

\begin{align}
\hat{\beta}_1 =& \frac{\Sigma_{i = 1}^{n} \left(x_i - \bar{x}\right)\left(y_i - \bar{y}\right)}
                    {\Sigma_{i = 1}^{n} \left(x_i - \bar{x}\right)^2} \nonumber \\
\hat{\beta}_0   =& \bar{y} - \hat{\beta}_1 \bar{x},
\end{align}

where \(\bar{x}\) and \(\bar{y}\) are means of \(x\) and \(y\),
respectively

\subsection{Assessing the Accuracy of the Coefficient
Estimates}\label{assessing-the-accuracy-of-the-coefficient-estimates}

True linear relationship between \(X\) and \(Y\)

\begin{align}
    Y = \beta_0 + \beta_1 X + \epsilon
\end{align}

The standard error (SE) of a statistic (usually an estimate of a
parameter) is the standard deviation of its sampling distribution or an
estimate of that standard deviation.

Standard error of a sample mean where \(\sigma^2 = \var(x)\) for a
population \(X\)

\begin{align}
    \var\left(\hat{\mu}\right) = \se\left(\hat{\mu}\right)^2 = \frac{\sigma^2}{n}
\end{align}

Standard error of estimates of linear coefficients where
\(\sigma^2 = \var(\epsilon)\)

\begin{align}
    \se\left(\hat{\beta}_0\right)^2 & = 
        \sigma^2\left[ 
            \frac{1}
                     {n} + 
            \frac{\bar{x}^2}
                     {\Sigma_{i - 1}^n\left(x_i - \bar{x}^2\right)} \right] 
            \nonumber \\
    \se\left(\hat{\beta}_1\right)^2 & = 
         \frac{\sigma{x}^2}
                    {\Sigma_{i - 1}^n \left(x_i - \bar{x}^2\right)}
\end{align}

There is approximately a 95\% chance that the true value of the slope
\(\beta_1\) lies in the interval

\begin{align}
    \left[  \hat{\beta}_1 - 2 \se\left( \hat{\beta}_1 \right),
                    \hat{\beta}_1 + 2 \se\left( \hat{\beta}_1 \right) 
                    \right].
\end{align}

There is approximately a 95\% chance that the true value of the
intercept \(\beta_0\) lies in the interval

\begin{align}
    \left[  \hat{\beta}_0 - 2 \se\left( \hat{\beta}_0 \right),
                    \hat{\beta}_0 + 2 \se\left( \hat{\beta}_0 \right) 
                    \right].
\end{align}

A common hypothesis test of linear relationships tests the null
hypothesis

\begin{align}
    H_0&: \text{There is no relationship between }X\text{ and }Y\nonumber \\
    H_0&: \beta_1 = 0
\end{align}

versus the alternate hypothesis

\begin{align}
    H_1&: \text{There is a relationship between }X\text{ and }Y\nonumber \\
    H_1&: \beta_1 \ne 0
\end{align}

using the \(t\)-statistic

\begin{align}
    t = \frac{\hat{\beta}_1 - 0}{\se\left(\hat{\beta}_1\right)},
\end{align}

which measures the number of standard deviations that \(\hat{\beta}_1\)
is away from the mean 0, because if there is no relationship between
\(X\) and \(Y\), then \(t\) will have a \(t\)-distribution with
\(df = n - 2\), that is, the degrees of freedom = the number of
variables \(-\) the number of parameters.

\subsection{Assessing the Accuracy of the
Model}\label{assessing-the-accuracy-of-the-model}

\subsubsection{Residual Standard Error}\label{residual-standard-error}

The \emph{residual standard error} is

\begin{align}
    \rse &= 
        \sqrt{\frac{1}{n-2}\rss}\nonumber \\
        &=\sqrt{\frac{1}{n-2}\Sigma_{i=1}^n\left(y_i - \hat{y}_i\right)^2} \\
        &\approx        \var(\epsilon)^2 \nonumber
\end{align}

where the \emph{residual sum of squares} is

\begin{align}
    \rss =\Sigma_{i=1}^n\left(y_i - \hat{y}_i\right)^2. 
\end{align}

\subsubsection{\texorpdfstring{\{\(R^2\)
Statistic\}}{\{R\^{}2 Statistic\}}}\label{r2-statistic}

The formula for \(R^2\) is

\begin{align}
    R^2 = \frac{\tss - \rss}{tss},
\end{align}

where the \emph{total sum of squares} is \[
\begin{align*}
    \tss =\Sigma_{i=1}^n\left(y_i - \bar{y}\right)^2. 
\end{align*}
\]

The \emph{correlation} of \(X\) and \(Y\) is

\begin{align}
r = \cor(X, Y) = \frac{\Sigma_{i = 1}^n\left(x_i - \bar{x}\right)\left(y_i - \bar{y}\right)}
        {\sqrt{\Sigma_{i = 1}^n\left(x_i - \bar{x}\right)}
                    \sqrt{\Sigma_{i = 1}^n\left(y_i - \bar{y}\right)}}
\end{align}

and in simple linear regression \[r^2 = R^2.\]


\end{document}
